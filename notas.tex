%%%%%%%%%%%%%%%%%%%%%%%%%%%%%%%%%%%%%%%%%
% Short Sectioned Assignment
% LaTeX Template
% Version 1.0 (5/5/12)
%
% This template has been downloaded from:
% http://www.LaTeXTemplates.com
%
% Original author:
% Frits Wenneker (http://www.howtotex.com)
%
% License:
% CC BY-NC-SA 3.0 (http://creativecommons.org/licenses/by-nc-sa/3.0/)
%
%%%%%%%%%%%%%%%%%%%%%%%%%%%%%%%%%%%%%%%%%

%----------------------------------------------------------------------------------------
%	PACKAGES AND OTHER DOCUMENT CONFIGURATIONS
%----------------------------------------------------------------------------------------

\documentclass[paper=a4, fontsize=11pt]{scrartcl} % A4 paper and 11pt font size
\usepackage[brazilian]{babel}
\usepackage[utf8]{inputenc}
\usepackage[T1]{fontenc}
\usepackage{amsmath,amsfonts,amsthm,mathtools} % Math packages
\usepackage{xspace}
\usepackage{indentfirst}

\newtheorem{theorem}{Teorema}
\newtheorem{definition}{Definição}
 

%\usepackage{sectsty} % Allows customizing section commands
%\allsectionsfont{\centering \normalfont\scshape} % Make all sections centered, the default font and small caps

\usepackage{fancyhdr} % Custom headers and footers
\pagestyle{fancyplain} % Makes all pages in the document conform to the custom headers and footers
\fancyhead{} % No page header - if you want one, create it in the same way as the footers below
\fancyfoot[L]{} % Empty left footer
\fancyfoot[C]{} % Empty center footer
\fancyfoot[R]{\thepage} % Page numbering for right footer
\renewcommand{\headrulewidth}{0pt} % Remove header underlines
\renewcommand{\footrulewidth}{0pt} % Remove footer underlines
\setlength{\headheight}{13.6pt} % Customize the height of the header

%\sectionfont{\bfseries\Large\raggedright}
%\subsectionfont{\bfseries\Large\raggedright}

%\numberwithin{equation}{section} % Number equations within sections (i.e. 1.1, 1.2, 2.1, 2.2 instead of 1, 2, 3, 4)
%\numberwithin{figure}{section} % Number figures within sections (i.e. 1.1, 1.2, 2.1, 2.2 instead of 1, 2, 3, 4)
%\numberwithin{table}{section} % Number tables within sections (i.e. 1.1, 1.2, 2.1, 2.2 instead of 1, 2, 3, 4)

%\setlength\parindent{0pt} % Removes all indentation from paragraphs - comment this line for an assignment with lots of text

%----------------------------------------------------------------------------------------
%	TITLE SECTION
%----------------------------------------------------------------------------------------

\newcommand{\horrule}[1]{\rule{\linewidth}{#1}} % Create horizontal rule command with 1 argument of height

\title{	
\normalfont \normalsize 
\textsc{DCC-IME-USP} \\ [25pt] % Your university, school and/or department name(s)
%\horrule{0.5pt} \\[0.4cm] % Thin top horizontal rule
\huge Complexidade de Contagem \\ % The assignment title
%\horrule{2pt} \\[0.5cm] % Thick bottom horizontal rule
}

%\renewcommand{\P}{P }
%\newcommand{\SP}{\#P }
%\newcommand{\NP}{NP }
%\newcommand{\FP}{FP }
%\newcommand{\PP}{PP }

\newcommand{\perm}{perm}

\renewcommand{\P}{\textbf{P}\xspace}
\newcommand{\SP}{\textbf{\#P}\xspace}
\newcommand{\NP}{\textbf{NP}\xspace}
\newcommand{\FP}{\textbf{FP}\xspace}
\newcommand{\PP}{\textbf{PP}\xspace}

\newcommand{\Pc}{\textbf{P-completo}}\xspace
\newcommand{\SPc}{\textbf{\#P-completo}\xspace}
\newcommand{\NPc}{\textbf{NP-completo}\xspace}
\newcommand{\FPc}{\textbf{FP-completo}\xspace}
\newcommand{\PPc}{\textbf{PP-completo}\xspace}

\newcommand{\prob}[1]{\textsc{\textbf{#1}}}


\author{Thales A. B. Paiva \\ thalespaiva@gmail.com} % Your name

\date{\normalsize\today} % Today's date or a custom date



\begin{document}


\maketitle % Print the title
\tableofcontents


%----------------------------------------------------------------------------------------
%	PROBLEM 1
%----------------------------------------------------------------------------------------

\pagebreak
\section{A classe \SP}

A classe \SP contém os problemas cuja solução é dada pela contagem de certificados de problemas em \NP.

\subsection{Exemplos de problemas em \SP}

\subsection{A classe \PP: $ \PP = \P \iff \SP = \FP $}

\pagebreak
\section{\SP-completude}

\subsection{Classes com oráculo}
\subsection{\prob{\#SAT} é \SPc}


\pagebreak
\section{Teorema de Valiant}

Valiant mostrou que calcular o número de emparelhamentos perfeitos num grafo bipartido é \SP-completo. Isso é surpreendente pois o problema de decisão associado está em \P. 

Para demonstrar esse teorema, primeiro lembramos da definição de permanente de uma matriz. Mostramos em seguida que, se considerarmos $A$ como uma matriz de adjacência de um digrafo $G$, calcular o permanente de $A$ é calcular o número de coberturas de ciclos de $G$. Finalmente, reduzimos o problema \#3SAT, que é \SPc, ao cálculo do número de coberturas de ciclos de um digrafo.

\subsection{Permanente}

\begin{definition} O permanente de uma matriz $A$ $n \times n$ é definido como: 
\[
\perm(A) = \sum\limits_{\sigma \in S_n} \left( \prod\limits_{i=1}a_{i,\sigma(i)} \right)
\]

\end{definition}

Calcular o permanente de uma matriz $n \times n$ pode ser interpretado como encontrar o número de emparelhamentos perfeitos num grafo $(A, B)$-bipartido em que A e B têm $n$ vértices cada.

\subsection{Cobertura por Ciclos}


\subsection{Demonstração}

\begin{theorem}[Valiant 1979] \prob{permanente} é \SP-completo.
\end{theorem}

\begin{proof}

Queremos reduzir \prob{\#3SAT} a \prob{permanente}. Para isso, dada uma fórmula $\phi$ na 3-FNC, devemos construir um digrafo $ G $ cujas coberturas de ciclos correspondam a valorações das variáveis de $\phi$ que a satisfazem.

Seja $\phi$ uma fórmula na forma normal conjuntiva com exatamente 3 variáveis por cláusula sobre as variáveis em $ X = \{x_1, \bar{x_1}, x_2, \bar{x_2} \dots, x_n, \bar{x_n} \} $. Assim, temos que
\begin{align*}
\phi = \bigwedge\limits_{i = 1}^{m}C_i = C_1 \land C_2 \land \dots \land C_m,
\end{align*}
onde cada cláusula $C_i$ é da forma $C_i = (y_{i1} \lor y_{i2} \lor y_{i3} )$, com cada $y_{ij}$ em $X$.

Iremos considerar três dispositivos para construir o digrafo $ G $ relacionado. Cada dispositivo representa alguma característica ou restrição de uma fórmula na 3-FNC em um digrafo. Os dispositivos e suas descrições resumidas são dadas a seguir.

\begin{description}
  \item[Dispositivo de Valoração $D_v$ :] \hfill \\
      Cada variável gera um dispositivo de valoração. Sua valoração está associada univocamente a como os dois vértices deste dispositivo são cobertos por uma cobertura de ciclos de $G$. A representação gráfica de $D_v$ para uma variável $x$ é:
      
      
      
  \item[Dispositivo de Cláusula:] \hfill \\
    sdf
  \item[Dispositivo de Valoração Exclusiva:] \hfill \\
    dsdf
\end{description}



\end{proof}



\subsection{Implicações}

\pagebreak
\section{Aproximações em \SP}

\pagebreak
\section{Teorema de Toda}


%----------------------------------------------------------------------------------------
\pagebreak
\begin{thebibliography}{1}

\bibitem{Val79a} Valiant, Leslie G. "The complexity of enumeration and reliability problems." SIAM Journal on Computing 8.3 (1979): 410-421.

\bibitem{Val79b} Valiant, Leslie G. "The complexity of computing the permanent." Theoretical computer science 8.2 (1979): 189-201.

\end{thebibliography}

\end{document}