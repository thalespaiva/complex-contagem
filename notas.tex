%%%%%%%%%%%%%%%%%%%%%%%%%%%%%%%%%%%%%%%%%
% Short Sectioned Assignment
% LaTeX Template
% Version 1.0 (5/5/12)
%
% This template has been downloaded from:
% http://www.LaTeXTemplates.com
%
% Original author:
% Frits Wenneker (http://www.howtotex.com)
%
% License:
% CC BY-NC-SA 3.0 (http://creativecommons.org/licenses/by-nc-sa/3.0/)
%
%%%%%%%%%%%%%%%%%%%%%%%%%%%%%%%%%%%%%%%%%

%----------------------------------------------------------------------------------------
%	PACKAGES AND OTHER DOCUMENT CONFIGURATIONS
%----------------------------------------------------------------------------------------

\documentclass[paper=a4, fontsize=11pt]{scrartcl} % A4 paper and 11pt font size
\usepackage[brazilian]{babel}
\usepackage[utf8]{inputenc}
\usepackage[T1]{fontenc}
\usepackage{amsmath,amsfonts,amsthm,mathtools} % Math packages
\usepackage{xspace}
\usepackage{indentfirst}
\usepackage{placeins}

\newtheorem{theorem}{Teorema}
\newtheorem{definition}{Definição}
 
\usepackage{tikz}
\usetikzlibrary{arrows}
\usetikzlibrary{positioning}
\usetikzlibrary{calc}

%\usepackage{sectsty} % Allows customizing section commands
%\allsectionsfont{\centering \normalfont\scshape} % Make all sections centered, the default font and small caps

\usepackage{fancyhdr} % Custom headers and footers
\pagestyle{fancyplain} % Makes all pages in the document conform to the custom headers and footers
\fancyhead{} % No page header - if you want one, create it in the same way as the footers below
\fancyfoot[L]{} % Empty left footer
\fancyfoot[C]{} % Empty center footer
\fancyfoot[R]{\thepage} % Page numbering for right footer
\renewcommand{\headrulewidth}{0pt} % Remove header underlines
\renewcommand{\footrulewidth}{0pt} % Remove footer underlines
\setlength{\headheight}{13.6pt} % Customize the height of the header

\bibliographystyle{apalike}

%\sectionfont{\bfseries\Large\raggedright}
%\subsectionfont{\bfseries\Large\raggedright}

\numberwithin{equation}{section} % Number equations within sections (i.e. 1.1, 1.2, 2.1, 2.2 instead of 1, 2, 3, 4)
\numberwithin{figure}{section} % Number figures within sections (i.e. 1.1, 1.2, 2.1, 2.2 instead of 1, 2, 3, 4)
\numberwithin{table}{section} % Number tables within sections (i.e. 1.1, 1.2, 2.1, 2.2 instead of 1, 2, 3, 4)
\numberwithin{definition}{section}
\numberwithin{theorem}{section}

%\setlength\parindent{0pt} % Removes all indentation from paragraphs - comment this line for an assignment with lots of text

%----------------------------------------------------------------------------------------
%	TITLE SECTION
%----------------------------------------------------------------------------------------

\newcommand{\horrule}[1]{\rule{\linewidth}{#1}} % Create horizontal rule command with 1 argument of height

\title{	
\normalfont \normalsize 
\textsc{DCC-IME-USP} \\ [25pt] % Your university, school and/or department name(s)
%\horrule{0.5pt} \\[0.4cm] % Thin top horizontal rule
\huge Complexidade de Contagem \\ % The assignment title
%\horrule{2pt} \\[0.5cm] % Thick bottom horizontal rule
}


\newcommand{\perm}{perm}
\newcommand{\words}{$\Sigma^*$\xspace}
\newcommand{\mwords}{\Sigma^*\xspace}

\renewcommand{\P}{\textbf{P}\xspace}
\newcommand{\SP}{\textbf{\#P}\xspace}
\newcommand{\NP}{\textbf{NP}\xspace}
\newcommand{\FP}{\textbf{FP}\xspace}
\newcommand{\PP}{\textbf{PP}\xspace}
\newcommand{\gP}{\textbf{GapP}\xspace}
\newcommand{\PSPACE}{\textbf{PSPACE}\xspace}

\newcommand{\Pc}{\textbf{P-completo}}\xspace
\newcommand{\SPc}{\textbf{\#P-completo}\xspace}
\newcommand{\NPc}{\textbf{NP-completo}\xspace}
\newcommand{\FPc}{\textbf{FP-completo}\xspace}
\newcommand{\PPc}{\textbf{PP-completo}\xspace}

\newcommand{\prob}[1]{\text{\textsc{\textbf{#1}}}}


\author{Thales A. B. Paiva \\ thalespaiva@gmail.com} % Your name

\date{\normalsize\today} % Today's date or a custom date



\begin{document}


\maketitle % Print the title
\tableofcontents


%----------------------------------------------------------------------------------------
%	PROBLEM 1
%----------------------------------------------------------------------------------------

\pagebreak
\section{Definições Preliminares}

Nesta seção, introduzimos algumas notações e definições. Foram extraídas principalmente de \cite{Fortnow97}.

Fixe $\Sigma = \{0, 1\}$ nosso alfabeto. Então $\Sigma^*$ é o conjunto de todas as palavras sobre esse alfabeto. $\Sigma^n$ é o conjunto de todas as strings de tamanho $n$ sobre esse alfabeto. 

\begin{definition}[Máquina NP] Uma \textbf{Máquina NP} é uma máquina de Turing não determinística cujo tempo de execução é limitado por um polinômio no tamanho da entrada.
\end{definition}

\begin{definition}[$\bar{M}$] Seja $M$ uma máquina NP, denotamos por \textbf{$\bar{M}$} a máquina NP que simula $M$ invertendo a decisão de $M$ sobre uma dada entrada.
\end{definition}

\begin{definition} Seja $M$ uma máquina NP, denotamos por $\#M(x)$ o número de caminhos de aceitação para uma entrada $x$, ou o número de certificados de que $x$ pertence à linguagem reconhecida por $M$. Fica claro, então, que $\#\bar{M}(x)$ é o número de caminhos de rejeição para uma entrada $x$.
\end{definition}

\begin{definition} Seja $M$ uma máquina NP, chamamos de \textbf{Intervalo} (\textit{Gap}) de M(x), denotado por $\Delta M(x)$, a diferença entre o número de caminhos de aceitação e de rejeição para uma entrada $x$. Então
\[
\Delta M(x) = \#M(x) - \#\bar{M}(x)
\]
\end{definition}

Note que, se o tempo de execução de $M$ é limitado pelo polinômio $|x|^k$ para cada entrada $x$, temos o tempo de execução de $\#M(x)$ e de $\Delta M(x)$ são limitados por $2^{|x|^k}$. Pois lembre que $2^{|x|^k}$ é o número de folhas de uma árvore de altura $|x|^k$.

\begin{definition} A classe de problemas de função \FP contém as funções que são computáveis em tempo polinomial.
\end{definition}

Note que \FP é a generalização da classe de problemas de decisão \P.



\pagebreak
\section{Classes de contagem}

Uma classe de contagem é um conjunto de problemas cujas soluções são dadas por funções sobre o número de certificados para problemas em NP. Nesta seção, definimos as duas principais classes de contagem, \SP e \gP. Jugamos que elas são as principais pois, podemos caracterizar outras classes de problemas a partir delas.

\subsection{\SP}

A classe \SP contém os problemas de função cujas soluções são dada pelo número de certificados de problemas em \NP. Em inglês, \SP é lido \textit{sharp}-P, mas para evitar anglicismos, sugerimos ler cerquilha-P ou número-P.

Podemos definir \SP facilmente em termos de uma máquina NP $M$ e $\#M$.
\begin{definition} A classe \SP contém as funções $f$ tais que existe uma máquina NP $M$ para que toda entrada $x \in \mwords$, temos $f(x) = \#M(x)$.
\end{definition}

Uma definição alternativa para \SP usando máquinas de Turing polinomiais, como  a de \cite{Arora09} é dada abaixo.
\begin{definition}[De \cite{Arora09}] Uma função $f:\mwords \rightarrow \mathbb{N}$ está em \SP se existe um polinômio $p:\mathbb{N} \rightarrow \mathbb{N}$ e uma máquina de Turing polinomial $M$ tal que, para todo $x \in \mwords$, tem-se que
$$
f(x) = \left| \{ y \in \{0, 1\}^{p(|x|)}: M(x, y) = 1 \} \right|
$$
\end{definition}

Como Arora aponta em \cite{Arora09}, a principal questão em aberto sobre \SP é se seus problemas podem ser resolvidos eficientemente ($ \SP \stackrel{?}{=} \FP $). É direto a definição que:
\begin{itemize}
  \item $ \SP = \FP \implies \NP = \P $.
  \item $ \SP \subset \PSPACE $.
  \item $ \P = \PSPACE \implies \P = \SP $.
\end{itemize}

Porém, não sabemos se $\NP = \P \stackrel{?}{\implies} \SP = \FP$.

Nem todo problema em 

\subsection{\gP}

Agora, definimos a classe GapP, que generaliza #P
A classe \gP consiste das 


\subsection{Caracterização da classe \PP}

\pagebreak
\section{\SP-completude}

\subsection{Classes com oráculo}
\subsection{Reduções parcimoniosas}
\subsection{\prob{\#SAT} é \SPc}


\pagebreak
\section{Teorema de Valiant}

Valiant mostrou que calcular o número de emparelhamentos perfeitos num grafo bipartido é \SP-completo. Isso é surpreendente pois o problema de decisão associado está em \P. 

Para demonstrar esse teorema, primeiro lembramos que calcular o número de emparelhamentos perfeitos num grafo bipartido é o mesmo que encontrar o número de coberturas de ciclos de um grafo direcionado. Mostramos em seguida que, se considerarmos $A$ como uma matriz de adjacência de um digrafo $G$, calcular o permanente de $A$ é calcular o número de coberturas de ciclos de $G$. Finalmente, reduzimos o problema \#3SAT, que é \SPc, ao cálculo do número de coberturas de ciclos de um digrafo.

\subsection{$ \prob{\#CycleCover} \equiv \prob{\#Matching}$}
\begin{proof}
Dado um grafo $G$, $(A, B)$-bipartido, queremos construir um digrafo $D$ cujo número de coberturas de ciclos seja igual ao número de emparelhamentos perfeitos de $G$. Suponha que $|A| = |B|$ e que
$$
A = \{a_1, a_2, \dots, a_n\}, \text{ e } \\
B = \{b_1, b_2, \dots, b_n\}.
$$

Se, $|A| \neq |B|$, não haveria emparelhamento perfeito e o digrafo vazio resolveria o problema. Construa $D = (A, E)$ em que:
$$
E = \{ (a_i, a_j) : a_i \in A, b_j \in \text{Adj}_G(a_i) \}.
$$

Se $M = \{(u_1, v_2), (u_2, v_2), \dots, (u_n, v_n) \} \subset A \times B$ for um emparelhamento perfeito, temos a seguinte cobertura de ciclos:
$$
C = \bigcup\limits_{i = 1}^n \{\\{ \}}
$$


\end{proof}

\subsection{$ \prob{Permanent} \equiv \prob{\#CycleCover} $}

\begin{definition} O permanente de uma matriz $A$ $n \times n$ é definido como: 
\[
\perm(A) = \sum\limits_{\sigma \in S_n} \left( \prod\limits_{i=1}a_{i,\sigma(i)} \right)
\]

\end{definition}

Calcular o permanente de uma matriz $n \times n$ pode ser interpretado como encontrar o número de emparelhamentos perfeitos num grafo $(A, B)$-bipartido em que A e B têm $n$ vértices cada.

\pagebreak
\subsection{Demonstração}

\begin{theorem}[Valiant 1979] \prob{permanente} é \SP-completo.
\end{theorem}

\begin{proof}

Queremos reduzir \prob{\#3sat} a \prob{permanente}. Para isso, dada uma fórmula $\phi$ na 3-FNC, devemos construir um digrafo $ G $ cujas coberturas de ciclos correspondam a valorações das variáveis de $\phi$ que a satisfazem.

Seja $\phi$ uma fórmula na forma normal conjuntiva com exatamente 3 variáveis por cláusula sobre as variáveis em $ X = \{x_1, \bar{x_1}, x_2, \bar{x_2} \dots, x_n, \bar{x_n} \} $. Assim, temos que
\begin{align*}
\phi = \bigwedge\limits_{i = 1}^{m}C_i = C_1 \land C_2 \land \dots \land C_m,
\end{align*}
onde cada cláusula $C_i$ é da forma $C_i = (y_{i1} \lor y_{i2} \lor y_{i3} )$, com cada $y_{ij}$ em $X$.

Iremos considerar três dispositivos para construir o digrafo $ G $ relacionado. Cada dispositivo representa alguma característica ou restrição de uma fórmula na 3-FNC em um digrafo. Os dispositivos e suas descrições resumidas são dadas a seguir.

\paragraph{Dispositivo de Valoração $D_V$ :} \hfill \\

      Cada variável gera um dispositivo de valoração. Sua valoração está associada univocamente a como os dois vértices deste dispositivo são cobertos por uma cobertura de ciclos de $G$. A representação gráfica de $D_v$ para uma variável $x$ pode ser vista na Figura~\ref{fig:Dv}, em que $0$ e $1$ representam respectivamente os valores verdadeiro e falso. Dois dos arcos estão pontilhados pois poderão ser acoplados a outros dispositivos na construção de $G$.
      
\FloatBarrier
\begin{figure}
\centering
\begin{tikzpicture}[->,>=stealth',auto,node distance=3cm,
  thick,main node/.style={circle,draw,font=\sffamily\Large\bfseries}]

  \node[main node] (1) { };
  \node[main node] (2) [below of=1] { };

  \path[every node/.style={font=\sffamily\small}]
    (1) edge node {} (2)
    (2) edge[dashed, bend right] node [right] {$x = 0$} (1)
    (2) edge[dashed, bend left] node [left] {$x = 1$} (1);
\end{tikzpicture}
\caption{Dispositivo de valoração para variável $x$}
\label{fig:Dv}
\end{figure}

\paragraph{Dispositivo de Cláusula $D_C:$} \hfill \\

Semelhante ao dispositivo anterior, cada cláusula gera um dispositivo de cláusula. Este dispositivo captura a restrição de que ao menos uma das expressões componentes de uma cláusula deve ser verdadeiro. Assim, para uma cláusula $C$, cada ciclo em seu dispositivo associado corresponde a uma valoração que faz $C$ ser satisfeita.

Nesse dispositivo, três arcos externos representam as valorações que fazem cada expressão de uma cláusula $C = (y_1 \lor y_2 \lor y_3)$ ser falsa. Podemos ver na Figura~\ref{fig:Dc} que não pode haver cobertura de ciclos que percorra os três arcos externos. Também vemos que para qualquer aresta externa ou par de arestas externas, há apenas uma cobertura de ciclos do dispositivo. Aqui, os arcos externos estão tracejados pois serão conectados a dispositivos de valoração através de dispositivos de valoração exclusiva.

\begin{figure}
\centering
\begin{tikzpicture}[->,>=stealth',auto,node distance=3cm,
  thick,main node/.style={circle,draw,font=\sffamily\Large\bfseries}]

  \node[main node] (1) { };
  \node[main node] (4) [below of=1] { };
  \node[main node] (2) [below left of=4] { };
  \node[main node] (3) [below right of=4] { };
  \path[every node/.style={font=\sffamily\small}]
    (1) edge[dashed, bend right] node [left]{$y_1 = 0$} (2)
    (2) edge[dashed, bend right] node [below]{$y_2 = 0$} (3)
    (3) edge[dashed, bend right] node [right]{$y_3 = 0$} (1)
    
    (1) edge[bend right=10] node {} (4)
    (4) edge[bend right=10] node {} (1)
    (2) edge[bend right=10] node {} (4)
    (4) edge[bend right=10] node {} (2)
    (3) edge[bend right=10] node {} (4)
    (4) edge[bend right=10] node {} (3)
    (2) edge[bend left=10] node {} (3)
    (3) edge[bend left=10] node {} (2)
%    (2) edge[dashed, bend right] node [right] { } (1)
%    (2) edge[dashed, bend left] node [left] { } (1);
;
\end{tikzpicture}
\caption{Dispositivo de cláusula para a cláusula $C = (y_1 \lor y_2 \lor y_3)$}
\label{fig:Dc}
\end{figure}
\FloatBarrier

\paragraph{Dispositivo de Valoração Exclusiva $D_\oplus$:} \hfill \\

Este dispositivo é o que garante a consistência entre os dispositivos de valoração e de cláusulas. Queremos garantir que uma cobertura de ciclos de $G$ não possua arestas que se contradizem. Ou seja, garantir que as cobertura dos dispositivos de valoração para cada variável de uma cláusula $C$ não entrem em conflito com as coberturas de $C$.

Como exemplo, suponha que a fórmula $\phi$ seja composta por apenas duas cláusulas e dada por

$$
\phi = C_1 \land C_2 = (x_1 \lor x_2 \lor \bar{x_3})\land(\bar{x_2} \lor x_3 \lor \bar{x_4}).
$$

Um esquema do digrafo para $ \phi $ é dado na Figura~\ref{fig:Doplus}. Nesta figura, os $D_\oplus$ representam os dispositivos de valoração exclusiva que devem ser acoplados às arestas contraditórias.

\begin{figure}
\centering
\begin{tikzpicture}[->,>=stealth',auto,node distance=2cm,
  thick,main node/.style={circle,draw,font=\sffamily\Large\bfseries}]
  
  \node[main node] (A1) [] { };
  \node[main node] (D1) [below of=A1] { };
  \node[main node] (B1) [below left of=D1] { };
  \node[main node] (C1) [below right of=D1] { };
  \path[every node/.style={font=\sffamily\small}]
    (A1) edge[dashed, bend right] node [left]{$x_1 = 0$} (B1)
    (B1) edge[dashed, bend right] node [below]{$x_2 = 0$} (C1)
    (C1) edge[dashed, bend right] node [right]{$\bar{x_3} = 0$} (A1)

    (A1) edge[bend right=10] node {} (D1)
    (D1) edge[bend right=10] node {} (A1)
    (B1) edge[bend right=10] node {} (D1)
    (D1) edge[bend right=10] node {} (B1)
    (C1) edge[bend right=10] node {} (D1)
    (D1) edge[bend right=10] node {} (C1)
    (B1) edge[bend left=10] node {} (C1)
    (C1) edge[bend left=10] node {} (B1);  
    
  \node[main node] (A2) [right=6 of A1] { };
  \node[main node] (D2) [below of=A2] { };
  \node[main node] (B2) [below left of=D2] { };
  \node[main node] (C2) [below right of=D2] { };
  \path[every node/.style={font=\sffamily\small}]
    (A2) edge[dashed, bend right] node [left]{$x_3 = 0$} (B2)
    (B2) edge[dashed, bend right] node [below]{$\bar{x_2} = 0$} (C2)
    (C2) edge[dashed, bend right] node [right]{$\bar{x_4} = 0$} (A2)

    (A2) edge[bend right=10] node {} (D2)
    (D2) edge[bend right=10] node {} (A2)
    (B2) edge[bend right=10] node {} (D2)
    (D2) edge[bend right=10] node {} (B2)
    (C2) edge[bend right=10] node {} (D2)
    (D2) edge[bend right=10] node {} (C2)
    (B2) edge[bend left=10] node {} (C2)
    (C2) edge[bend left=10] node {} (B2);  \node[main node] (A2) { };

--
  \node[main node] (2) [left of=B1] { };
  \node[main node] (1) [above of=2]{ };
  \path[every node/.style={font=\sffamily\small}]
    (1) edge node {} (2)
    (2) edge[dashed, bend right] node [right] {$x_1 = 1$} (1)
    (2) edge[bend left] node [left] {$x_1 = 0$} (1);

  %\draw node[] (Dx) [below right=0.1 and 0.1 of 1] {$D_\oplus$};

  \node[main node] (3) [right of=C2] { };
  \node[main node] (4) [above of=3]{ };
  \path[every node/.style={font=\sffamily\small}]
    (4) edge node {} (3)
    (3) edge[dashed, bend left] node [left] {$x_4 = 0$} (4)
    (3) edge[bend right] node [right] {$x_4 = 1$} (4);
    

  \node[main node] (8) at ($(C1)!0.5!(B2)$) { }; %[left of=B2]{ };
  \node[main node] (7) [below of=8] { };
  \path[every node/.style={font=\sffamily\small}]
    (8) edge node {} (7)
    (7) edge[dashed, bend left] node [left] {$x_2 = 1$} (8)
    (7) edge[dashed, bend right] node [right] {$x_2 = 0$} (8);

  \node[main node] (5) [above of=8] {};%at ($(A2)!0.5!(A1)$) { } ;%[right of=A1] { };
  \node[main node] (6) [above of=5]{ };
  \path[every node/.style={font=\sffamily\small}]
    (6) edge node {} (5)
    (5) edge[dashed, bend left] node [left] {$x_3 = 0$} (6)
    (5) edge[dashed, bend right] node [right] {$x_3 = 1$} (6);

  \draw node[circle] [below right=0.1 and 0.7 of 1] {$D_\oplus$};

  \draw node[circle] [below left=0.1 and 0.7 of 4] {$D_\oplus$};
 
  \draw node[circle] [above right=0.00 and 0.7 of 5] {$D_\oplus$};
  \draw node[circle] [above left=0.00 and 0.7 of 5] {$D_\oplus$};

  \draw node[circle] [above right=0.5 and 1.6 of 7] {$D_\oplus$};
  \draw node[circle] [above left=0.5 and 1.6 of 7] {$D_\oplus$};
  
\end{tikzpicture}
\caption{$D_\oplus$ para forçar o uso exclusivo de arestas contraditórias}
\label{fig:Doplus}
\end{figure}



Gostaríamos que $D_\oplus$ garantisse que exatamente uma das arestas contraditórias fosse atravessada. Porém, se isso fosse possível, esta redução seria parcimoniosa, o que implicaria $ \P = \NP$. Pois como sabemos dizer se $ \perm(A) \geq 0 $ eficientemente, saberíamos dizer se uma instância de \prob{3sat} tem ao menos uma solução.

Assim, usaremos um digrafo com pesos nas arestas para desconsiderar coberturas de ciclos de $G$ que implicariam valorações contraditórias. O dispositivo resultante pode ser visto na Figura~\ref{fig:DoplusDef}. Sua matriz de adjacências $ M_{D_\oplus} $ é dada a seguir.

\begin{align*}
M_{D_\oplus} = 
\begin{bmatrix}
0 &  1 & -1 & -1  \\
1 & -1 &  1 &  1  \\
0 &  1 &  1 &  2  \\
0 &  1 &  3 &  0  \\
\end{bmatrix}
\end{align*}

\begin{figure}
\centering
\begin{tikzpicture}[->,>=stealth',auto,node distance=2.5cm,
  thick,main node/.style={circle,draw,font=\sffamily\small\bfseries}]

  \node[main node] (a) {a};
  \node[main node] (b) [above right of=a] {b};
  \node[main node] (d) [below right of=b] {d};
  \node[main node] (c) [below right of=a] {c};

  \path[every node/.style={font=\sffamily\small}]
    (a) edge[bend left=10] node {} (b)
    (a) edge[] node [pos=0.2, below]  {-1} (d)    
    (a) edge[bend right=10] node [below] {-1} (c)
    
    (b) edge[bend left=10] node {} (a)
    (b) edge[loop above] node {-1} (b)
    (b) edge[bend left=10] node {} (c)
    (b) edge[bend left=10] node {} (d)

    (c) edge[bend left=10] node {} (b)
    (c) edge[bend left=10] node [above] {2} (d)
    (c) edge[loop below] node [above] {} (c)    
    
    (d) edge[bend left=10] node [above] {} (b)
    (d) edge[bend left=10] node [below] {3} (c)    
    ;

  \node[main node] (u1) [above left of=a]{$u_1$};
  \node[main node] (u2) [above right of=d] {$u_2$};
  \node[main node] (v1) [below right of=d] {$v_1$};
  \node[main node] (v2) [below left of=a] {$v_2$};

  \path[every node/.style={font=\sffamily\small}]
    (u1) edge[bend left=10] node {} (a)
    (d) edge[bend left=10] node {} (u2)  
    (v1) edge[bend left=10] node {} (d)
    (a) edge[bend left=10] node {} (v2)  
    ;  
\end{tikzpicture}
\caption{O dispositivo de valoração exclusiva $D_\oplus$ sobre arcos $(u_1, u_2)$ e $(v_1, v_2)$}
\label{fig:DoplusDef}
\end{figure}
\FloatBarrier





%  \node[main node] (D1) [below right of=2] {$D_\oplus$};






\end{proof}




\subsection{Implicações}

\pagebreak
\section{Aproximações em \SP}

\pagebreak
\section{Teorema de Toda}


%----------------------------------------------------------------------------------------
\pagebreak
\begin{thebibliography}{1}

\bibitem{Val79a} Valiant, Leslie G. "The complexity of enumeration and reliability problems." SIAM Journal on Computing 8.3 (1979): 410-421.

\bibitem{Val79b} Valiant, Leslie G. "The complexity of computing the permanent." Theoretical computer science 8.2 (1979): 189-201.

\bibitem{Arora09} Arora, Sanjeev, and Boaz Barak. Computational complexity: a modern approach. Cambridge University Press, 2009.

\bibitem{Fortnow97} Fortnow, Lance. "Counting complexity." Complexity theory retrospective II (1997): 81-107.

\end{thebibliography}

\end{document}